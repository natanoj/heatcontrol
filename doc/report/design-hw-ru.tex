% Copyright Jonatan Åkerlind 2012
% design process of the room unit

\subsubsection{Component selection}
The unit should present a few parameters to the user and also be able
to react on user input. For this requirement a simple LCD and a few
push buttons is sufficient. A simple two row LCD with integrated
driver, the Newhaven Display NHD-C0220BIZ-FS(RGB) with integrated RGB
backlight, is selected. For the push buttons debounce an MCU is
used. This MCU will also be used to drive the RGB backlight diodes
using PWM to control the color and brightness. The Atmel ATtiny24A is
chosen since it has three PWM output channels and enough I/O to
support the push buttons.

Communications by Ethernet need connector, magnetic components for
electrical isolation and an Ethernet interface circuit. The MAG45
Ethernet jack with integrated magnetic, LED and PoE diode components
is chosen to simplify design effort and minimize board size. For the
driver circuit the Micrel KSZ8851 is selected due to price and simple
SPI interface. For Power over Ethernet operations a DC-DC converter
transformer with a PWM driver and a PoE driver circuit is needed. The
Texas Instruments TPS23757 is selected as the driver circuit since it
has both PoE and DC-DC PWM converter driving interfaces. Würth
eletronics has several options for ready-made flyback DC-DC PWM
transformers with PoE level input, auxilary output for driving the PWM
circuit matching the TPS23757 and output options for wanted Vdd
voltage.

As environment sensors the Microchip TC1046 is chosen as the
temperature sensor due to price as well as being pre-calibrated. The
Honeywell HIH-5031 is chosen as the relative humidity sensor since it
has a very simple linear voltage output which will simplify the sensor
interface design process. An optional pyrolytic infrared (PIR) sensor
can be used to detect human presence in the room that is
regulated. The Zilog ZMOTION Detection Module II is chosen for
simplicity since the module is ready to use with a simple serial
interface and the relatively small form factor.

The Atmel ATmega1284P is chosen as the MCU since it has I/O that will
support SPI interface for the Ethernet controller and TWI interface
for the display and ATtiny24A, AD converters for temperature and
humidity sensors. Four interrupt inputs are used for pushbutton
events, ethernet data events, ethernet power management events (Wake
on LAN) and PIR signal events.

\subsubsection{Power budget}
Power budget for all components except the PSU components can be seen
in \tabref{tab:powerbudget_roomunit}. All components can or need to be
driven by 3.3 V which therefore is selected as the overall system
voltage, $V_{DD}$. The total maximum power budget for the design is
seen in the table as approximately 140 mA. For the power supply output
stage a maximum current of 250 mA and a minimum of 100 mA is chosen.

\begin{table}
\begin{tabular}{|l|l|l|l|}
\hline
Reference designator & Name        & Typical current & Max current\\
\hline
U100                 & TC1046      & 35 $\mu$A        & 60 $\mu$A\\
U101                 & HIH-5031    & 200 $\mu$A       & 500 $\mu$A\\
A100                 & ZDM II      & $<$ 8.9 mA          & 8.9 mA\\
U200                 & NHD-C0220BIZ-FS & 1 mA         & 1 mA ? \\
U200 (LED)           & NHD-C0220BIZ-FS(RGB) & $<$ 44 mA  & 44 mA \\
U201                 & ATtiny24A   & 210 $\mu$A       & 210 $\mu$A ? \\
U301                 & KSZ8851     & 85 mA            & 85 mA ? \\
U400                 & ATMega1284P & 0.4 mA           & 0.4 mA ? \\
D552                 & LGQ971      & 20 mA            & 20 mA \\
\hline
Total                &             & $<$ 160 mA            & 160 mA \\
\hline
\end{tabular}
\caption{Power budget for room unit 3.3 V components}
\label{tab:powerbudget_roomunit}
\end{table}


\subsubsection{PoE converter}
According to the power budget summary
in \tabref{tab:powerbudget_roomunit} and the previous section the
maximum power drawn is approximately 330 mW. This is without the power
consumed by the DC/DC converter itself (TPS23757). According to the
PoE standard there are different classifications based upon maximum
average power for a powered device. The lowest power class (Class 1)
specifies a minimum power of 440 mW and a maximum power of 3.84 W. If
the power consumption drops below the minimum the power source
equipment may turn the PoE off and this needs to be verified
carefully. The class is chosen by a classification resistor connected
to the TPS23757 (R501 in the schematics) and should be 243 $\Omega$
for Class 1.

The optional DC adapter interface uses a resistor devider network
(R505, R506) to get the proper voltage at the APD pin of the
TPS23757. This will be designed to be used with a 48 V DC $\pm$10 \%
adapter according to the examples given in Texas Instruments document
SLVA306A. R506 is chosen as 6.8 k$\Omega$ and the minimum startup
voltage is chosen 25 \% below 48 V

\begin{equation}
Vin_{MIN}=48 * 0.75 = 36 \text{ V}
\end{equation}
The ratio between $Vin_{MIN}$ and adapter priority pin typical voltage,
$V_{APDEN}=1.5\text{V}$, is given by
\begin{equation}
DR_{APD}=\frac{Vin_{MIN}}{V_{APDEN}}=\frac{36 \text{ V}}{1.5 \text{ V}}=24
\end{equation}
From here the R505 can be calculated by
\begin{equation}
R505 = R506 * (DR_{APD} - 1) = 156.40 \text{ k$\Omega$}
\end{equation}
R505 is chosen as 158 k$\Omega$. Now the adapter turnon voltage
becomes
\begin{equation}
Vin_{MIN}=\frac{R505+R506}{R506} * V_{APDEN}
= \frac{164.8 \text{ k$\Omega$}}{6.8 \text{ k$\Omega$}} * 1.5 \text{ V}
= 36.353 \text{ V}
\end{equation}
and the adapter turnoff voltage, $V_{ADPTR\_OFF}$, is given by the
minimum adapter priority pin voltage ($V_{APDEN} - V_{APDH}$), where
$V_{APDH}$ is the hysteresis for the APD pin which is 0.31 V
typically. 
\begin{equation}
V_{ADPTR\_OFF}=\frac{R505+R506}{R506} * (V_{APDEN} - V_{APDH})
= \frac{164.8 \text{ k$\Omega$}}{6.8 \text{ k$\Omega$}} * 1.2 \text{ V} =
29.082 \text{ V}
\end{equation}
The maximum voltage at the APD pin is, using a 48 V adapter with 10 \% error
\begin{equation}
V_{APD} < \frac{Vin_{MAX}}{\frac{R505+R506}{R506}}
= \frac{48*1.1 \text{ V}}{\frac{164.8 \text{ k$\Omega$}}{6.8 \text{ k$\Omega$}}}
= 2.179 \text{ V}
\end{equation}
which is less than $V_B$ (5 V) and so within the recommended maximum
voltage.

The output stage consists of a PoE transformer (T550) with input (48 V),
output (3.3 V) and auxilary (12 V) windings. The Würth Electronics
part number 7491199331 is chosen, which is designed for 300 kHz
switching frequency, has 127 $\mu$H input inductance, 3.5 $\mu$H
leakage inductance and 1.5 $kV_{AC}$ breakdown between windings. To
program the TPS23757 to run at 300 kHz a resistor is connected between
the FRS and ARTN pins (R502) with a value given by
\begin{equation}
R502 = \frac{17250}{f_{SW} (kHz)} = \frac{17250}{300} \text{
k$\Omega$} = 57.5 \text{ k$\Omega$}
\end{equation}
A resistor with 57.6 k$\Omega$ resistance is chosen. The transformer
is specified for input voltages between 36 - 57 V DC and has a winding
ratio (n) between input and output windings of 12.5.

With an assumed converter efficiency ($\eta$) of 80 \% the switching
transistor Q550 and the rectifying diode D551 will be exposed to
voltage and current stresses according the following calculations.

\begin{equation}
V_{Imin}=36 \text{ V}, 
\end{equation}
\begin{equation}
V_{Imax}=57 \text{ V}
\end{equation}
Minimum DC voltage transfer function
\begin{equation}
M_{V DCmin}=\frac{V_O}{V_{Imax}}=\frac{3.3}{57}=0.057895
\end{equation}
and maximum DC voltage transfer function
\begin{equation}
M_{V DCmax}=\frac{V_O}{V_{Imin}}=\frac{3.3}{36}=0.091667
\end{equation}
Minimum and maximum duty cycles
\begin{equation}
D_{min}=\frac{nM_{V DCmin}}{nM_{V DCmin}+\eta}=\frac{12.5*0.058}{12.5*0.058+0.8}=0.475
\end{equation}
\begin{equation}
D_{max}=\frac{nM_{V DCmax}}{nM_{V DCmax}+\eta}=\frac{12.5*0.092}{12.5*0.092+0.8}=0.589
\end{equation}
Which is within the TPS23757 duty cycle range of 0 - 78 \%. The
peak-to-peak ac current through the magnetizing inductance (input) is
\begin{equation}
\Delta i_{Lm(max)}=\frac{nV_O (1-D_{min})}{f_s L_m}
= \frac{12.5*3.3*(1-0.475)}{3*10^5*127*10^{-6}} \text{ A} = 0.568 \text{ A}
\end{equation}
The maximum input DC current occurs at the maximum DC voltage transfer
function (minimum DC input voltage)
\begin{equation}
I_{Imax}=M_{V DCmax}I_{Omax} = 0.0917 * 0.250 \text{ A} =
0.0229 \text{ A}
\end{equation}
This gives the current and voltage stresses of the switching
transistor and the rectifying diode
\begin{equation}
I_{SMmax} = I_{Imax} + \frac{I_{Omax}}{n} + \frac{\Delta
i_{Lm(max)}}{2} = 0.0229 + \frac{0.250}{12.5} + \frac{0.568}{2} \text{
A} = 
0.327 \text{ A}
\end{equation}
\begin{equation}
I_{DMmax} = nI_{Imax} + I_{Omax} + \frac{n\Delta i_{Lm(max)}}{2} = 
12.5*0.0229 + 0.250 + \frac{12.5*0.568}{2} \text{ A} = 4.089 \text{ A}
\end{equation}
\begin{equation}
V_{SMmax} = V_{Imax} +nV_O = 57 + 12.5*3.3 \text{ V} = 98.25 \text{ V}
\end{equation}
\begin{equation}
V_{DMmax} = \frac{V_{Imax}}{n} + V_O = \frac{57}{12.5} + 3.3 \text{ V}
= 7.86 \text{ V}
\end{equation}
The N-MOSFET IRFL4315PbF from International Rectifier can handle
$V_{DSS}=150$ V and $I_D=2.6$ A with a maximum $R_{DS(on)}$ of 185
m$\Omega$ at $V_{GS}=10$ V. Typical gate charge $Q_g=12$ nC (max 19
nC) and output capacitance $C_O=98$ pF. NXP PMEG3050BEP Schottky diode
has a maximum $I_{DM}=5$ A, maximum reverse voltage $V_{DM}=30$ V and
forward voltage drop $V_F=0.3$ V at 0.3 A which gives an equivalent
resistance $R_F$ of 1 $\Omega$. Output voltage ripple should be below
1 \%
\begin{equation}
V_r=0.01V_O = 0.01*3.3 \text{ V} = 33 \text{ mV}
\end{equation}

For the current sense input to the control circuit a resistor,
$R_{CS}$ (R551), is chosen from the primary winding peak current and
max input voltage on $V_{CS}$. According to the datasheet $V_{CSMAX}$
= 0.55 V.

\begin{equation}
R_{cs_{MIN}} = \frac{V_{CSMAX}}{I_{SMmax}} = \frac{0.55}{0.327} =
1.682 \text{\Omega}
\end{equation}
$R_{CS}$ is chosen as 1.8 $\Omega$. Maximum power dissipation for this
would then be approx 0.19 W if the peak current was constant. This
corner case is highly unlikely but dimensioning for this case will
keep the design on the safe side. An SMD resistor rated for 250 mW is
chosen.

% with only one filter capacitor
Equivalent minimum load resistance
\begin{equation}
R_{Lmin}=\frac{V_O}{I_{Omax}} = \frac{3.3}{0.25} \text{ $\Omega$} =
13.2 \text{ $\Omega$}
\end{equation}

Assuming ripple voltage over filter capacitor ESR $V_{rcpp}=27$ mV and
the rest over the filter capacitance $V_{Cpp}=6$ mV gives the maximum
ESR of the filter capacitor
\begin{equation}
r_{Cmax} = \frac{V_{rcpp}}{I_{DMmax}} =
\frac{27}{4.089} \text{ m$\Omega$} = 6.6 \text{ m$\Omega$}
\end{equation}
and the filter capacitance
\begin{equation}
C_{min} = \frac{D_{max}V_O}{f_sR_{Lmin}V_{Cpp}} \text{ F} =
\frac{0.589*3.3}{3*10^5*13.2*6*10^{-3}} = 81.8 \text{ $\mu$F}
\end{equation}


Output filter in a C-L-C configuration with values derived from the
previous section and verified with simulation. Component values chosen
are 68 nF for the first shunt capacitor (C552 toward the transformer), 50
nH for the inductor (L550) and 22 \text{ $\mu$F} for the second shunt
capacitor (C553 toward the regulated power output).
% FIXME:  Add information and simulation graph


Output voltage set point resistors are calculated using the estimated
integrator zero location at approximately 1 kHz. Integrator zero
capacitors range between 1 nF - 10 nF, 4.7 nF is assumed.
\begin{equation}
R_{IZE} \approx \frac{1}{2* \pi * 4.7 nF * 1 kHz} =
33.9 \text{k$\Omega$}
\end{equation}
\begin{equation}
R_{FBUE} \approx R_{IZE}
\end{equation}
$R_{FBU}$ (R553) chosen as 41.2 \text{k$\Omega$}. $V_{ref}$ = 1.24 V
which then gives $R_{FBL}$ (R554)
\begin{equation}
R_{FBL} = \frac{V_{ref}*R_{FBU}}{V_{out_{NOM}} - V_{ref}}
= \frac{1.24*41.2}{3.3-1.24} = 24.8 \text{k$\Omega$}
\end{equation}
$R_{FBL}$ chosen at 24.3 \text{k$\Omega$} which gives a slightly
higher nominal output voltage. This will compensate a small voltage drop
in the output circuit.

Opto-Isolator, U550, should have a current-transfer-ratio of about
80-160 \% at 5 mA. The Fairchild FOD817A is chosen. At 1-2 mA LED bias
current it has an approximate current-transfer-ratio of 85 \%. Forward
LED voltage $V_{led_{OPTO}}$ at 2 mA is approximately 1.1 V. Bias
current limiting resistor, $R_{OB}$ (R552) calculated at zero duty
cycle
\begin{equation}
V_{led{OPTO_K}} = V_{ref} + 150 mV
\end{equation}
\begin{equation}
R_{OB} = \frac{V_{out_{NOM}} - V_{led_{OPTO}} -
V_{led_{OPTO-K}}}{I_{led_{OPTO}}} = \frac{3.3 - 1.1 - 1.25}{2*10^{-3}}
= 475 \Omega
\end{equation}
The opto-isolator transistor bias current calculated at zero duty
cycle (values from the datasheet of TPS23757). $V_{zdc_{MAX}}$ = 1.7
V, $V_{cs_{MAX}}$ = 0.55 V at $V_B$ = 5 V.
\begin{equation}
V_{ctl_{MAX}} = V_{zdc_{MAX}} + 2*V_{cs_{MAX}}
\end{equation}
\begin{equation}
V_{ctl_{NOM}} = \frac{V_{zdc_{max}} + V_{ctl_{MAX}}}{2} = 2.25 V
\end{equation}
\begin{equation}
R_{CTL} = \frac{V_b - V_{zdc_{MAX}}}{I_{led_{OPTO}}*CTR_{2mA}} =
1.9412 \text{k$\Omega$}
\end{equation}
$R_{CTL}$ (R550) chosen as 2 \text{k$\Omega$}.

% FIXME: do we need secondary side soft start according to SLVA305C 2.8.6 ?


\subsubsection{Peripheral components}
The display needs two capacitors to drive the internal boost converter
for 5 V supply, these are 1 $\mu$F each. The backlight LEDs are
specified according to \tabref{tab:backlightleds} and need serial
resistors with values according to \ref{eq:rledred}, \ref{eq:rledgreen}
and \ref{eq:rledblue}.
\begin{equation}
R_{Red} = \frac{3.3 - 1.9}{17} \text{ k$\Omega$} = 82 \text{ $\Omega$}
\label{eq:rledred}
\end{equation}
\begin{equation}
R_{Green} = \frac{3.3 - 3.0}{17} \text{ k$\Omega$} = 18 \text{ $\Omega$}
\label{eq:rledgreen}
\end{equation}
\begin{equation}
R_{Blue} = \frac{3.3 - 3.0}{10} \text{ k$\Omega$} = 30 \text{ $\Omega$}
\label{eq:rledblue}
\end{equation}

A number of digital nets need pull-up or pull-down resistors, to
minimize leakage currents a typical value of 100 k$\Omega$ is
used. This will only draw 33 $\mu$A when the line is being driven. To
keep external components around the microcontrollers at a minimum the
internal pull-ups are enabled for unused pins and these are left
unconnected. This has the drawback of higher power consumption during
reset condition, but this only occurs either when programming the
microcontroller program memories or when a user presses the reset
button. The latter should however be seen as an exception and will
probably only be used during software development or hardware
verification.


\begin{table}
\begin{tabular}{|l|l|l|}
\hline
Colour & Voltage drop (V) & Current (mA)\\
\hline
Red & 1.9 & 17\\
Green & 3.0 & 17\\
Blue & 3.0 & 10\\
\hline
\end{tabular}
\caption{LCD Backlight LED drive voltages and currents}
\label{tab:backlightleds}
\end{table}
