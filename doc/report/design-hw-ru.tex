% Copyright Jonatan Åkerlind 2012
% design process of the room unit

\subsubsection{Component selection}
The unit should present a few parameters to the user and also be able
to react on user input. For this requirement a simple LCD and a few
push buttons is sufficient. A simple two row LCD with integrated
driver, the Newhaven Display NHD-C0220BIZ-FS(RGB) with integrated RGB
backlight, is selected. For the push buttons debounce an MCU is
used. This MCU will also be used to drive the RGB backlight diodes
using PWM to control the color and brightness. The Atmel ATtiny24A is
chosen since it has three PWM output channels and enough I/O to
support the push buttons.

Communications by Ethernet need connector, magnetic components for
electrical isolation and an Ethernet interface circuit. The MAG45
Ethernet jack with integrated magnetic, LED and PoE diode components
is chosen to simplify design effort and minimize board size. For the
driver circuit the Micrel KSZ8851 is selected due to price and simple
SPI interface. For Power over Ethernet operations a DC-DC converter
transformer with a PWM driver and a PoE driver circuit is needed. The
Texas Instruments TPS23757 is selected as the driver circuit since it
has both PoE and DC-DC PWM converter driving interfaces. Würth
eletronics has several options for ready-made flyback DC-DC PWM
transformers with PoE level input, auxilary output for driving the PWM
circuit matching the TPS23757 and output options for wanted Vdd
voltage.

As environment sensors the Microchip TC1046 is chosen as the
temperature sensor due to price as well as being pre-calibrated. The
Honeywell HIH-5031 is chosen as the relative humidity sensor since it
has a very simple linear voltage output which will simplify the sensor
interface design process. An optional pyrolytic infrared (PIR) sensor
can be used to detect human presence in the room that is
regulated. The Zilog ZMOTION Detection Module II is chosen for
simplicity since the module is ready to use with a simple serial
interface and the relatively small form factor.

The Atmel ATmega1284P is chosen as the MCU since it has I/O that will
support SPI interface for the Ethernet controller and TWI interface
for the display and ATtiny24A, AD converters for temperature and
humidity sensors. Four interrupt inputs are used for pushbutton
events, ethernet data events, ethernet power management events (Wake
on LAN) and PIR signal events.

All resistors are chosen from the E48 series.

\subsubsection{Power budget}
Power budget for all components except the PSU components can be seen
in \tabref{tab:powerbudget_roomunit}. All components can or need to be
driven by 3.3 V which therefore is selected as the overall system
voltage, $V_{DD}$. The total maximum power budget for the design is
seen in the table as approximately 140 mA. For the power supply output
stage a maximum current of 250 mA and a minimum of 100 mA is chosen.

\begin{table}
\begin{tabular}{|l|l|l|l|}
\hline
Reference designator & Name        & Typical current & Max current\\
\hline
U100                 & TC1046      & 35 $\mu$A        & 60 $\mu$A\\
U101                 & HIH-5031    & 200 $\mu$A       & 500 $\mu$A\\
A100                 & ZDM II      & $<$ 8.9 mA          & 8.9 mA\\
U200                 & NHD-C0220BIZ-FS & 1 mA         & 1 mA ? \\
U200 (LED)           & NHD-C0220BIZ-FS(RGB) & $<$ 44 mA  & 44 mA \\
U201                 & ATtiny24A   & 210 $\mu$A       & 210 $\mu$A ? \\
U301                 & KSZ8851     & 85 mA            & 85 mA ? \\
U400                 & ATMega1284P & 0.4 mA           & 0.4 mA ? \\
D552                 & LGQ971      & 20 mA            & 20 mA \\
\hline
Total                &             & $<$ 160 mA            & 160 mA \\
\hline
\end{tabular}
\caption{Power budget for room unit 3.3 V components}
\label{tab:powerbudget_roomunit}
\end{table}


\subsubsection{PoE converter}
According to the power budget summary
in \tabref{tab:powerbudget_roomunit} and the previous section the
maximum power drawn is approximately 330 mW. This is without the power
consumed by the DC/DC converter itself (TPS23757). According to the
PoE standard there are different classifications based upon maximum
average power for a powered device. The lowest power class (Class 1)
specifies a minimum power of 440 mW and a maximum power of 3.84 W. If
the power consumption drops below the minimum the power source
equipment may turn the PoE off and this needs to be verified
carefully. The class is chosen by a classification resistor connected
to the TPS23757 (R501 in the schematics) and should be 243 $\Omega$
for Class 1.

The optional DC adapter interface uses a resistor devider network
(R505, R506) to get the proper voltage at the APD pin of the
TPS23757. This will be designed to be used with a 48 V DC $\pm$10 \%
adapter according to the examples given in Texas Instruments document
SLVA305A. R506 is chosen as 6.8 k$\Omega$ and the minimum startup
voltage is chosen 25 \% below 48 V
% FIXME: reference slva305a properly

\begin{equation}
Vin_{MIN}=48 * 0.75 = 36 \text{ V}
\end{equation}
The ratio between $Vin_{MIN}$ and adapter priority pin typical voltage,
$V_{APDEN}=1.5\text{V}$, is given by
\begin{equation}
DR_{APD}=\frac{Vin_{MIN}}{V_{APDEN}}=\frac{36 \text{ V}}{1.5 \text{ V}}=24
\end{equation}
From here the R505 can be calculated by
\begin{equation}
R505 = R506 * (DR_{APD} - 1) = 156.40 \text{ k$\Omega$}
\end{equation}
R505 is chosen as 158 k$\Omega$. Now the adapter turnon voltage
becomes
\begin{equation}
Vin_{MIN}=\frac{R505+R506}{R506} * V_{APDEN}
= \frac{164.8 \text{ k$\Omega$}}{6.8 \text{ k$\Omega$}} * 1.5 \text{ V}
= 36.353 \text{ V}
\end{equation}
and the adapter turnoff voltage, $V_{ADPTR\_OFF}$, is given by the
minimum adapter priority pin voltage ($V_{APDEN} - V_{APDH}$), where
$V_{APDH}$ is the hysteresis for the APD pin which is 0.31 V
typically.
\begin{equation}
V_{ADPTR\_OFF}=\frac{R505+R506}{R506} * (V_{APDEN} - V_{APDH})
= \frac{164.8 \text{ k$\Omega$}}{6.8 \text{ k$\Omega$}} * 1.2 \text{ V} =
29.082 \text{ V}
\end{equation}
The maximum voltage at the APD pin is, using a 48 V adapter with 10 \% error
\begin{equation}
V_{APD} < \frac{Vin_{MAX}}{\frac{R505+R506}{R506}}
= \frac{48*1.1 \text{ V}}{\frac{164.8 \text{ k$\Omega$}}{6.8 \text{ k$\Omega$}}}
= 2.179 \text{ V}
\end{equation}
which is less than $V_B$ (5 V) and so within the recommended maximum
voltage.

The flyback modulator power stage with feedback network is designed
according to the example given in SLV305A.
% FIXME: reference SLV305A
To minimize the transformer size Discontinous Conduction Mode is
used. With an assumed converter efficiency ($\eta$) of 85 \% and the
given DC voltage transfer ratios according to

\begin{equation}
V_{Imin}=36 \text{ V},
\end{equation}
\begin{equation}
V_{Imax}=57 \text{ V}
\end{equation}
Minimum DC voltage transfer function
\begin{equation}
M_{V DCmin}=\frac{V_O}{V_{Imax}}=\frac{3.3}{57}=0.057895
\end{equation}
and maximum DC voltage transfer function
\begin{equation}
M_{V DCmax}=\frac{V_O}{V_{Imin}}=\frac{3.3}{36}=0.091667
\end{equation}

and assumed duty cycle at DCM/CCM boundary $D_{b max}$=0.3 the minimum
input-output winding ratio of the transformer becomes

\begin{equation}
N_{ps min}=\eta \frac{D_{b max}}{(1-D_{b max})M_{V DCmax}}=3.97
\end{equation}

The PoE transformer (T550) with input (48 V), output (3.3 V and 5 V)
and auxilary (12 V) windings. Prebuilt transformer from TDK part
number EFD15 B82802A0012A315 is chosen, which has 100 $\mu$H input
inductance, 3 $\mu$H leakage inductance at 100 kHz and 1.5 $kV_{AC}$
breakdown between windings. The transformer is specified for input
voltages between 36 - 60 V DC and has a winding ratio (n) between
input and primary output windings of 10. Chosen switching frequency is
200 kHz, to program the TPS23757 to run at 200 kHz a resistor is
connected between the FRS and ARTN pins (R502) with a value given by
\begin{equation}
R502 = \frac{17250}{f_{SW} (kHz)} = \frac{17250}{200} \text{
k$\Omega$} = 86.25 \text{ k$\Omega$}
\end{equation}

A resistor with 86.6 k$\Omega$ resistance is chosen.

To keep the flyback in DCM the maximum magnetizing inductance is
\begin{equation}
L_{m max}=frac{N_{ps}^2 R_{l min} (1-D_{b max})^2}{2
f_{s}}=489 \text{$\mu$ H}
\end{equation}

Which is above the chosen transformer rating.

The switching transistor Q550 and the rectifying diode D551 will be
exposed to voltage and current stresses according the following
calculations.

Minimum and maximum duty cycles
\begin{equation}
D_{min}=M_{V DCmin} \sqrt{\frac{2 f_s L_m}{\eta R_{l min}}}=0.10931
\end{equation}
\begin{equation}
D_{max}=M_{V DCmax} \sqrt{\frac{2 f_s L_m}{\eta R_{l min}}}=0.17308
\end{equation}

%% only for CCM(?)
The peak-to-peak ac current through the magnetizing inductance (input) is
\begin{equation}
\Delta i_{Lm(max)}=\frac{nV_O (1-D_{min})}{f_s L_m}
= \frac{10*3.3*(1-0.10931)}{2*10^5*100*10^{-6}} \text{ A} = 1.4696 \text{ A}
\end{equation}

The current and voltage stresses of the switching transistor and the
rectifying diode are given by
\begin{equation}
I_{SMmax} = \frac{D_{min}V_{Imax}}{f_s L_m} = \frac{0.109*57}{200*10^3*100*10^{-6}} \text{
A} = 0.31154 \text{ A}
\end{equation}
\begin{equation}
I_{DMmax} = nI_{SMmax} = 10*0.31154 \text{ A} = 3.1154 \text{ A}
\end{equation}
\begin{equation}
V_{SMmax} = V_{Imax} +nV_O = 57 + 10*3.3 \text{ V} = 90 \text{ V}
\end{equation}
\begin{equation}
V_{DMmax} = \frac{V_{Imax}}{n} + V_O = \frac{57}{10} + 3.3 \text{ V}
= 9 \text{ V}
\end{equation}
The N-MOSFET IRLM0100TRPbF from International Rectifier can handle
$V_{DSS}=100$ V and $I_D=1.6$ A with a maximum $R_{DS(on)}$ of 220
m$\Omega$ at $V_{GS}=10$ V. Typical gate charge $Q_g=2.5$ nC and
output capacitance $C_O=27$ pF. Multicomp SS54 Schottky diode has a
maximum $I_{DM}=5$ A, maximum reverse voltage $V_{DM}=40$ V and
forward voltage drop $V_F=0.55$ V at 5 A which gives an equivalent
resistance $R_F$ of 0.11 $\Omega$.

Adjusted peak voltage at switching transistor due to transformer leakage
inductance $L_l$, primary $R_p$ and secondary $R_s$ winding
resistances, stray input capacitance $C_p$, and transistor output
capacitance $C_O$. Ringing frequency
\begin{equation}
f_{ring}=\frac{\omega_{ring}}{2\pi}=\frac{1}{\sqrt{L_l(C_O+C_p)}}
\frac{1}{2\pi}=15.106 \text{ MHz}
\end{equation}
characteristic impedance of ringing circuit
\begin{equation}
Z_0=\sqrt{\frac{L_l}{C_O+C_p}}=284.75 \Omega
\end{equation}
ringing voltage spike
\begin{equation}
V_{spike}=I_{SMmax}Z_0=88.711 \text{ V}
\end{equation}
adjusted switching voltage stress
\begin{equation}
V_{SMmax adj}=V_{Imax}+nV_O+V_{spike}=178.71 \text{ V}
\end{equation}
This will be limited with snubber circuit $D_{sn}$ (D553), $R_{sn}$
(R558), and $C_{sn}$ (C555) with $V_{spike new}$ as 5 V.
\begin{equation}
C_{sn(min)}=(\frac{V_{spike}}{V_{spike
new}})^2(C_O+C_p)=11.647 \text{ nF}
\end{equation}
Choosing $C_{sn}$ as 15 nF. Choose $R_{sn}$ for time constant much
larger than switching frequency
\begin{equation}
R_{sn}=\frac{200}{f_s C_{sn}}=66.667 \text{ k$\Omega$}
\end{equation}
Choosing $R_{sn}$ as 64.9 k$\Omega$.
For $D_{sn}$ the MURA120 from ON Semiconductor is chosen, with peak
reverse voltage $V_R$ 200 V and average forward current $I_F$ 1 A.

Maximum input voltage ripple targeted below 1 V gives minimum filter
capacitance
\begin{equation}
C_{in(min)}=\frac{\Delta i_{Lm(max)}D_{max}}{f_s V_{Iripple}}=
1.2718 \mu F
\end{equation}
Choosing $C_{in}$ as one low ESR X7R MLCC 2.2 $\mu$F (C556) and one bulk aluminum
electrolytic at 2.2 $\mu$F (C550).


For the current sense input to the control circuit a resistor,
$R_{CS}$ (R551), is chosen from the primary winding peak current and
max input voltage on $V_{CS}$. According to the datasheet $V_{CSMAX}$
= 0.55 V.

\begin{equation}
R_{cs_{MIN}} = \frac{V_{CSMAX}}{I_{SMmax}} = \frac{0.55}{0.327} =
1.682 \text{$\Omega$}
\end{equation}
$R_{CS}$ is chosen as 1.8 $\Omega$. Maximum power dissipation for this
would then be approx 0.19 W if the peak current was constant. This
corner case is highly unlikely but dimensioning for this case will
keep the design on the safe side. An SMD resistor rated for 250 mW is
chosen.


Output voltage ripple should be below 2 \%
\begin{equation}
V_r=0.02V_O = 0.02*3.3 \text{ V} = 66 \text{ mV}
\end{equation}

% with only one filter capacitor
Equivalent minimum load resistance
\begin{equation}
R_{Lmin}=\frac{V_O}{I_{Omax}} = \frac{3.3}{0.25} \text{ $\Omega$} =
13.2 \text{ $\Omega$}
\end{equation}

Assuming ripple voltage over filter capacitor ESR $V_{rcpp}=27$ mV and
the rest over the filter capacitance $V_{Cpp}=6$ mV gives the maximum
ESR of the filter capacitor
\begin{equation}
r_{Cmax} = \frac{V_{rcpp}}{I_{DMmax}} =
\frac{27}{4.089} \text{ m$\Omega$} = 6.6 \text{ m$\Omega$}
\end{equation}
and the filter capacitance
\begin{equation}
C_{min} = \frac{D_{max}V_O}{f_sR_{Lmin}V_{Cpp}} \text{ F} =
\frac{0.589*3.3}{3*10^5*13.2*6*10^{-3}} = 81.8 \text{ $\mu$F}
\end{equation}


Output voltage set point resistors are calculated using the estimated
integrator zero location at approximately 1 kHz. Integrator zero
capacitors range between 1 nF - 10 nF, 4.7 nF is assumed.
\begin{equation}
R_{IZE} \approx \frac{1}{2* \pi * 4.7 nF * 1 kHz} =
33.9 \text{k$\Omega$}
\end{equation}
\begin{equation}
R_{FBUE} \approx R_{IZE}
\end{equation}
$R_{FBU}$ (R553) chosen as 41.2 \text{k$\Omega$}. $V_{ref}$ = 1.24 V
which then gives $R_{FBL}$ (R554)
\begin{equation}
R_{FBL} = \frac{V_{ref}*R_{FBU}}{V_{out_{NOM}} - V_{ref}}
= \frac{1.24*41.2}{3.3-1.24} = 24.8 \text{k$\Omega$}
\end{equation}
$R_{FBL}$ chosen at 24.3 \text{k$\Omega$} which gives a slightly
higher nominal output voltage. This will compensate a small voltage drop
in the output circuit.

Opto-Isolator, U550, should have a current-transfer-ratio of about
80-160 \% at 5 mA. The Fairchild FOD817A is chosen. At 1-2 mA LED bias
current it has an approximate current-transfer-ratio of 85 \%. Forward
LED voltage $V_{led_{OPTO}}$ at 2 mA is approximately 1.1 V. Bias
current limiting resistor, $R_{OB}$ (R552) calculated at zero duty
cycle
\begin{equation}
V_{led{OPTO_K}} = V_{ref} + 150 mV
\end{equation}
\begin{equation}
R_{OB} = \frac{V_{out_{NOM}} - V_{led_{OPTO}} -
V_{led_{OPTO-K}}}{I_{led_{OPTO}}} = \frac{3.3 - 1.1 - 1.25}{2*10^{-3}}
= 475 \Omega
\end{equation}
The opto-isolator transistor bias current calculated at zero duty
cycle (values from the datasheet of TPS23757). $V_{zdc_{MAX}}$ = 1.7
V, $V_{cs_{MAX}}$ = 0.55 V at $V_B$ = 5 V.
\begin{equation}
V_{ctl_{MAX}} = V_{zdc_{MAX}} + 2*V_{cs_{MAX}}
\end{equation}
\begin{equation}
V_{ctl_{NOM}} = \frac{V_{zdc_{max}} + V_{ctl_{MAX}}}{2} = 2.25 V
\end{equation}
\begin{equation}
R_{CTL} = \frac{V_b - V_{zdc_{MAX}}}{I_{led_{OPTO}}*CTR_{2mA}} =
1.9412 \text{k$\Omega$}
\end{equation}
$R_{CTL}$ (R550) chosen as 2 \text{k$\Omega$}.

% FIXME: do we need secondary side soft start according to SLVA305C 2.8.6 ?


\subsubsection{Peripheral components}
The display needs two capacitors to drive the internal boost converter
for 5 V supply, these are 1 $\mu$F each. The backlight LEDs are
specified according to \tabref{tab:backlightleds} and need serial
resistors with values according to \ref{eq:rledred}, \ref{eq:rledgreen}
and \ref{eq:rledblue}.
\begin{equation}
R_{Red} = \frac{3.3 - 1.9}{17} \text{ k$\Omega$} = 82 \text{ $\Omega$}
\label{eq:rledred}
\end{equation}
\begin{equation}
R_{Green} = \frac{3.3 - 3.0}{17} \text{ k$\Omega$} = 18 \text{ $\Omega$}
\label{eq:rledgreen}
\end{equation}
\begin{equation}
R_{Blue} = \frac{3.3 - 3.0}{10} \text{ k$\Omega$} = 30 \text{ $\Omega$}
\label{eq:rledblue}
\end{equation}

A number of digital nets need pull-up or pull-down resistors, to
minimize leakage currents a typical value of 100 k$\Omega$ is
used. This will only draw 33 $\mu$A when the line is being driven. To
keep external components around the microcontrollers at a minimum the
internal pull-ups are enabled for unused pins and these are left
unconnected. This has the drawback of higher power consumption during
reset condition, but this only occurs either when programming the
microcontroller program memories or when a user presses the reset
button. The latter should however be seen as an exception and will
probably only be used during software development or hardware
verification.


\begin{table}
\begin{tabular}{|l|l|l|}
\hline
Colour & Voltage drop (V) & Current (mA)\\
\hline
Red & 1.9 & 17\\
Green & 3.0 & 17\\
Blue & 3.0 & 10\\
\hline
\end{tabular}
\caption{LCD Backlight LED drive voltages and currents}
\label{tab:backlightleds}
\end{table}


\subsubsection{Mechanical aspects}
The PCBs need to fit in the housing with display, buttons, and PIR
sensor at the proper height. The chosen buttons, Multicomp MCDTS2-4N
12 mm, are 4.3 mm from bottom edge (mounted to PCB) to the button
level. Switch caps, Multicomp KTSC-22, are in total 7.7 mm in height
with a stop surface at 3.5 mm from the bottom. This means that the
casing front inner plane should be at least $4.3+3.5=7.8$ mm from the
top layer of the PCB. The caps have a rounded outer edge at the top
with 1 mm radius and travelling length of 0.5 mm. Assuming a casing
thickness of 2 mm, the casing front inner plane can be at most
$4.3+7.7-1-0.5-2=8.5$ mm from the top layer of the PCB.

The PIR sensor, Panasonic PaPIR EKMC1603113, is in total 20.2 mm from
bottom edge (mounted to PCB) to the top of the lens. The lens is
approximately a half sphere with diameter 22 mm, but edges cut down to
a cylindar with 20.7 mm diameter for mounting in casing. The height
from top of the sphere to the cut edge is given by the circle segment
\begin{equation}
h(2r-h)=\left(\frac{x}{2}\right)^2
\end{equation}
where h is the wanted height, r the radius of the sphere, and x the
segment length. The resulting height is
\begin{equation}
h=\frac{2r}{2}\pm\sqrt{\left(-\frac{2r}{2}\right)^2-\left(\frac{x}{2}\right)^2}=
\begin{cases}
14.725 & >r ~\text{not wanted solution},\\
7.2747 & <r ~\text{wanted solution}.
\end{cases}
\end{equation}

The PIR sensor casing has an edge at 7.3+2 mm from the bottom, which
represents the lowest possible height from top of PCB to inner edge of
casing front.

Combining these dimensions the PIR sensor and the push buttons can be
directly mounted to the PCB with a 2 mm thick casing front at 9.3 mm
from the PCB to inner edge of casing. With the push buttons mounted
closely to the PCB the caps will only protrude $\approx 0.7$ mm,
however the push buttons have 3.5 mm long leads and can therefore be
mounted with a $\approx 1$ mm distance to the PCB surface placing the
switch cap tops at $\approx 1.7$ mm above the casing outer surface.

The EA DOG-M display is 2 mm thick with 7.5 mm long leads at the
bottom for a total length of 9.5 mm from the top surface to the lead
ends. This means that the display needs to be mounted with some
special care. A separate board to mount the display and related
passives and cutting the display leads down to 2.5 mm would leave
approximately $9.3-2-2.5=4.8$ mm between top of PCB to display lead
ends.
