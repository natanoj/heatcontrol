% -*- Mode: latex; Mode: auto-fill -*-
The system should regulate the temperature of rooms in a house to
maintain a preset temperature and at the same time be conservative
with available energy. Temperature preset should be adjustable for
each room in a simple way from the rooms. Further the system should
have a way to set different temperatures during different time ranges
of one day, i.e. to lower the room temperature during night or when
the room is unused.

Another functionality of the system should be the ability of measuring
warm water temperature and controlling warm water circulation. The
temperature should be measured where the warm water circulation end
point is located, i.e. in the house which warm water is delivered
to. When this temperature drops below some preset value the warm water
circulation pump should be activated and run until the temperature
reaches the stored warm water temperature with heat losses calculated
for. 

\subsection{Regulating room temperature}
Room temperature can be controlled by adjusting
heat flow in a water based heating system. Each room will have either
water radiator elements or water based floor heating. The temperature
of the water system can be controlled in a number of ways:
\begin{itemize}
\item{Adjusting the temperature of water flowing in to the radiators
  and floor heating. These two systems can have individual water
  temperatures\footnote{The temperature of the floor heating water
    cannot exceed the temperature of the water radiators since the
    shunt for the floor heating takes warm water from the radiator
    heat water forward pipe} controlled by motorized shunt
  valves\footnote{The motorized shunt valves will have end limit
    switches, normally open non-potential contacts, indicating max or
    min setting}.}
\item{Controlling water flow in the floor heating system for each
  room. Water flow is controlled by electromagnetic valves which are
  either on or off.}
\item{Controlling water flow in the radiators for each room by turning
  an electromagnetic valve on or off. For systems with thermostats already
  installed there should be an option of indicating to the user if the
  thermostat should be adjusted up or down, i.e. an indirect way of
  controlling water flow in the radiators.}
\item{Circulation pumps for the two different water flows can be
  turned off when the system no longer needs to add heat and likewise
  turned on again for supplying heat.}
\end{itemize}

Heat supply for the system is stored in water accumulators charged by
a system outside the scope of this system, however it should not be
assumed that heat is always available in the accumulators. To reliably
regulate supplied temperature to the water circulation the temperature
of water stored in the accumulators should be measured. 

Input parameters for the regulation needs to be a number of
parameters:
\begin{itemize}
\item{Room temperature for each room.}
\item{Water radiator and floor heating forward and
  return\footnote{The return water temperature can be used to
    calculate total power dissipation for the system.} temperatures.}
\item{Heat water accumulator top and bottom temperatures. This is for
  measuring available energy.}
\item{Outdoor temperature should be used to compensate for energy
  leakage rising with lower outdoor temperatures.}
\end{itemize}
Additionally the system should measure relative air humidity both
outdoor and for each room. This should then be used to both add a
comfort factor, i.e. optionally raise the temperature if it is cold
and humid outside, as well as protect cold rooms from being too
humid. 


\subsection{Heat system interaction}
The interaction with a heater system is to be located at several
different physical locations. One is where the water accumulators are
located, i.e. the energy supply and storage which typically will be a
furnace, solar heater or some kind of heat pump which charges water
accumulators. Another interface is at one or several places where
floor heat water flow is controlled. Both of these types should have
the possibility to control electric shunting valves but only the
latter needs to control regular electric valves.

Regulating electric shunting valves includes checking that the valve
has not reached its end limit in the direction it is being turned. It
also includes measuring the temperature of the regulated water pipe,
which should be measured closely to the shunting valve, as well as the
return water temperature. The unit placed at the location of the water
accumulators needs to be able to measure one outdoor temperature, four
pipe temperatures (water radiator forward and return, accumulator top
and bottom temperatures) and one outdoor air relative humidity. It
also needs to be able to control two circulation pumps, one for warm
water circulation and one for heat water circulation. This heat water
circulation is to be considered a general purpose pump that can be
either heat water circulation over to the house that is to be heated
if the water accumulators are located at another location outside of
the house, or it can be circulation for a separate heater system. In
the first case it should be activated if either of the other
circulation pumps (floor heating or water radiator) are running and in
the second case it should be activated if a room unit for this
separate system requires more heat. In either case it is to be
considered optional.

Floor heating control unit needs to be able to control floor heating
electromagnetic valves for 12 rooms as well as two circulation pumps,
one for the water radiator circulation and one for floor heat
circulation. It should also be able to measure one air temperature and
one air relative humidity as well as the warm water pipe temperature
at the end point of the warm water circulation.

The system should have a standby mode when it should not apply heat
but keep the circulation pumps running for one hour each month. The
shunt valves should also be turned one complete turn from low to high
once each month. This is to keep the pumps and shunt valves from
jamming when the system is unused for a long period of time. All
electromagnetic valves should also be toggled once each month for the
same reason.


\subsection{System interfaces}
Human interaction with the system should be possible in a number of
different ways. In each room the system should display, without the
need to press any buttons or other manual interaction, current room
temperature, wanted room temperature, outdoor temperature and the
current day/night setting. In each room it should also be possible to
see the current room and outdoor relative humidity, however this may
require some manual interaction from the user. The wanted room
temperature as well as day/night setting for that room should be
adjustable from each room in a simple way. Day versus night setting
should be selectable as one of the following; follow the system time
schedule, override as day or night until time schedule changes next
time or override as day or night until user selects another
setting. If using the option of controlling water radiators indirectly
the system must indicate if water radiator thermostats should be
adjusted up or down. The room units controlling floor heating should
also indicate if heat is applied to the room or not.

The unit placed at the location of the water accumulators (the main
unit) should indicate if each output is on or off, if each shunt valve
is regulated up, down or kept at its current setting and display each
input value. It should also show current system day/night setting and
be able to show the day/night time schedule. All system parameters
except individual room temperatures should also be adjustable from
this unit. These are the specified system parameters:
\begin{itemize}
\item{Parameters for the regulation of the forward heat water
  temperatures, i.e. the dependency on outdoor and indoor
  temperatures, outdoor and indoor relative air humidity or manually
  raising or lowering the forward heat water temperature}
\item{Circulation pumps on or off}
\item{Current day or night setting}
\item{Time schedule for day or night setting} 
\item{Warm water circulation high and low temperatures}
\item{System on or standby}
\end{itemize}

Time schedule for day or night setting should have a granularity of 15
minutes, i.e. it should be possible to set a new day or night setting
once every 15 minutes. A finer granularity than this should not be
necessary since the system as a whole is slow; it takes some time to
effectuate a new temperature in a house. 

Floor heat controlling unit should indicate if each output is on or
off, indicate if each shunt valve is regulated up, down or kept at its
current setting and display each input value. It should also show
current system day/night setting. 

There should also be web access to the main unit, i.e. the main unit
should serve HTTP
%% FIXME: reference RFC for HTTP
requests, which should display all system parameters and current input
values. It should also be possible to change the system parameters by
using this HTTP access. Parameters and input values should be
presented in a way that will make it easy to read these parameters
from some other computer system for logging purposes. Optionally the
main unit should have an SNMP
%% FIXME: reference RFC for SNMP
server that will show all parameters and input values and also enable
setting of parameters. Preferably the system should also have an
option to log all or a selection of events to a syslog
%% FIXME: reference RFC for syslog
server.


\subsection{Unit communication}
Communication between the room units and the other units of the system
is to be carried over Internet protocol version 6 (IPv6) or version 4
(IPv4)
%% FIXME: reference RFCs for IPv4 and IPv6
, version should be configurable by the user. IP communication
will be carried over Ethernet
%% FIXME: reference IEEE802.3 for Ethernet
and power supply for the room units
should be Power over Ethernet (PoE)
%% FIXME: reference IEEE802.3af for PoE
. This will then enable the room
units to use a pre-installed Ethernet network in the house that is to
be regulated. Optionally the room units should have power input for
external power supply in case PoE is not available.

Main unit will naturally act as a common communication central for all
other units. All parameters should be sent to the main unit and stored
there in a way that will keep them even if power fails. 
